% Options for packages loaded elsewhere
\PassOptionsToPackage{unicode}{hyperref}
\PassOptionsToPackage{hyphens}{url}
%
\documentclass[
]{article}
\title{Lab 7}
\author{Your Name and UID go here}
\date{2022-05-22}

\usepackage{amsmath,amssymb}
\usepackage{lmodern}
\usepackage{iftex}
\ifPDFTeX
  \usepackage[T1]{fontenc}
  \usepackage[utf8]{inputenc}
  \usepackage{textcomp} % provide euro and other symbols
\else % if luatex or xetex
  \usepackage{unicode-math}
  \defaultfontfeatures{Scale=MatchLowercase}
  \defaultfontfeatures[\rmfamily]{Ligatures=TeX,Scale=1}
\fi
% Use upquote if available, for straight quotes in verbatim environments
\IfFileExists{upquote.sty}{\usepackage{upquote}}{}
\IfFileExists{microtype.sty}{% use microtype if available
  \usepackage[]{microtype}
  \UseMicrotypeSet[protrusion]{basicmath} % disable protrusion for tt fonts
}{}
\makeatletter
\@ifundefined{KOMAClassName}{% if non-KOMA class
  \IfFileExists{parskip.sty}{%
    \usepackage{parskip}
  }{% else
    \setlength{\parindent}{0pt}
    \setlength{\parskip}{6pt plus 2pt minus 1pt}}
}{% if KOMA class
  \KOMAoptions{parskip=half}}
\makeatother
\usepackage{xcolor}
\IfFileExists{xurl.sty}{\usepackage{xurl}}{} % add URL line breaks if available
\IfFileExists{bookmark.sty}{\usepackage{bookmark}}{\usepackage{hyperref}}
\hypersetup{
  pdftitle={Lab 7},
  pdfauthor={Your Name and UID go here},
  hidelinks,
  pdfcreator={LaTeX via pandoc}}
\urlstyle{same} % disable monospaced font for URLs
\usepackage[margin=1in]{geometry}
\usepackage{color}
\usepackage{fancyvrb}
\newcommand{\VerbBar}{|}
\newcommand{\VERB}{\Verb[commandchars=\\\{\}]}
\DefineVerbatimEnvironment{Highlighting}{Verbatim}{commandchars=\\\{\}}
% Add ',fontsize=\small' for more characters per line
\usepackage{framed}
\definecolor{shadecolor}{RGB}{248,248,248}
\newenvironment{Shaded}{\begin{snugshade}}{\end{snugshade}}
\newcommand{\AlertTok}[1]{\textcolor[rgb]{0.94,0.16,0.16}{#1}}
\newcommand{\AnnotationTok}[1]{\textcolor[rgb]{0.56,0.35,0.01}{\textbf{\textit{#1}}}}
\newcommand{\AttributeTok}[1]{\textcolor[rgb]{0.77,0.63,0.00}{#1}}
\newcommand{\BaseNTok}[1]{\textcolor[rgb]{0.00,0.00,0.81}{#1}}
\newcommand{\BuiltInTok}[1]{#1}
\newcommand{\CharTok}[1]{\textcolor[rgb]{0.31,0.60,0.02}{#1}}
\newcommand{\CommentTok}[1]{\textcolor[rgb]{0.56,0.35,0.01}{\textit{#1}}}
\newcommand{\CommentVarTok}[1]{\textcolor[rgb]{0.56,0.35,0.01}{\textbf{\textit{#1}}}}
\newcommand{\ConstantTok}[1]{\textcolor[rgb]{0.00,0.00,0.00}{#1}}
\newcommand{\ControlFlowTok}[1]{\textcolor[rgb]{0.13,0.29,0.53}{\textbf{#1}}}
\newcommand{\DataTypeTok}[1]{\textcolor[rgb]{0.13,0.29,0.53}{#1}}
\newcommand{\DecValTok}[1]{\textcolor[rgb]{0.00,0.00,0.81}{#1}}
\newcommand{\DocumentationTok}[1]{\textcolor[rgb]{0.56,0.35,0.01}{\textbf{\textit{#1}}}}
\newcommand{\ErrorTok}[1]{\textcolor[rgb]{0.64,0.00,0.00}{\textbf{#1}}}
\newcommand{\ExtensionTok}[1]{#1}
\newcommand{\FloatTok}[1]{\textcolor[rgb]{0.00,0.00,0.81}{#1}}
\newcommand{\FunctionTok}[1]{\textcolor[rgb]{0.00,0.00,0.00}{#1}}
\newcommand{\ImportTok}[1]{#1}
\newcommand{\InformationTok}[1]{\textcolor[rgb]{0.56,0.35,0.01}{\textbf{\textit{#1}}}}
\newcommand{\KeywordTok}[1]{\textcolor[rgb]{0.13,0.29,0.53}{\textbf{#1}}}
\newcommand{\NormalTok}[1]{#1}
\newcommand{\OperatorTok}[1]{\textcolor[rgb]{0.81,0.36,0.00}{\textbf{#1}}}
\newcommand{\OtherTok}[1]{\textcolor[rgb]{0.56,0.35,0.01}{#1}}
\newcommand{\PreprocessorTok}[1]{\textcolor[rgb]{0.56,0.35,0.01}{\textit{#1}}}
\newcommand{\RegionMarkerTok}[1]{#1}
\newcommand{\SpecialCharTok}[1]{\textcolor[rgb]{0.00,0.00,0.00}{#1}}
\newcommand{\SpecialStringTok}[1]{\textcolor[rgb]{0.31,0.60,0.02}{#1}}
\newcommand{\StringTok}[1]{\textcolor[rgb]{0.31,0.60,0.02}{#1}}
\newcommand{\VariableTok}[1]{\textcolor[rgb]{0.00,0.00,0.00}{#1}}
\newcommand{\VerbatimStringTok}[1]{\textcolor[rgb]{0.31,0.60,0.02}{#1}}
\newcommand{\WarningTok}[1]{\textcolor[rgb]{0.56,0.35,0.01}{\textbf{\textit{#1}}}}
\usepackage{graphicx}
\makeatletter
\def\maxwidth{\ifdim\Gin@nat@width>\linewidth\linewidth\else\Gin@nat@width\fi}
\def\maxheight{\ifdim\Gin@nat@height>\textheight\textheight\else\Gin@nat@height\fi}
\makeatother
% Scale images if necessary, so that they will not overflow the page
% margins by default, and it is still possible to overwrite the defaults
% using explicit options in \includegraphics[width, height, ...]{}
\setkeys{Gin}{width=\maxwidth,height=\maxheight,keepaspectratio}
% Set default figure placement to htbp
\makeatletter
\def\fps@figure{htbp}
\makeatother
\setlength{\emergencystretch}{3em} % prevent overfull lines
\providecommand{\tightlist}{%
  \setlength{\itemsep}{0pt}\setlength{\parskip}{0pt}}
\setcounter{secnumdepth}{-\maxdimen} % remove section numbering
\ifLuaTeX
  \usepackage{selnolig}  % disable illegal ligatures
\fi

\begin{document}
\maketitle

{
\setcounter{tocdepth}{3}
\tableofcontents
}
\fontsize{10}{11}

~~

\begin{verbatim}
## Date last run: 2022-05-22
\end{verbatim}

\begin{verbatim}
## Hello World!
\end{verbatim}

~~

\hypertarget{examples}{%
\section{Examples}\label{examples}}

Requires library xtable.

~~

\hypertarget{nwsl-2017-2021-season-team-game-data}{%
\subsection{NWSL 2017-2021 Season Team-Game
Data}\label{nwsl-2017-2021-season-team-game-data}}

~~

\begin{Shaded}
\begin{Highlighting}[]
\DocumentationTok{\#\# Read in our data}
\NormalTok{xdf }\OtherTok{\textless{}{-}} \FunctionTok{read.table}\NormalTok{(}\StringTok{"NWSL\_gameTeam.tsv"}\NormalTok{, }\AttributeTok{sep=}\StringTok{"}\SpecialCharTok{\textbackslash{}t}\StringTok{"}\NormalTok{, }\AttributeTok{header=}\ConstantTok{TRUE}\NormalTok{)}

\FunctionTok{head}\NormalTok{(xdf, }\AttributeTok{n=}\DecValTok{6}\NormalTok{)}
\end{Highlighting}
\end{Shaded}

\begin{verbatim}
##   season     date                     gameLeague                   team HT_VT Min Gls Ast
## 1   2017 20170415 National Women's Soccer League      Washington Spirit    HT 990   0   0
## 2   2017 20170415 National Women's Soccer League North Carolina Courage    VT 990   1   1
## 3   2017 20170415 National Women's Soccer League     Portland Thorns FC    HT 990   2   1
## 4   2017 20170415 National Women's Soccer League          Orlando Pride    VT 991   0   0
## 5   2017 20170415 National Women's Soccer League           Houston Dash    HT 990   2   2
## 6   2017 20170415 National Women's Soccer League      Chicago Red Stars    VT 990   0   0
##   PK PKatt CrdY CrdR
## 1  0     0    2    0
## 2  0     0    0    0
## 3  1     1    0    0
## 4  0     0    0    0
## 5  0     0    1    0
## 6  0     0    1    0
\end{verbatim}

\begin{Shaded}
\begin{Highlighting}[]
\DocumentationTok{\#\#\# table(xdf[ , "team"])}
\end{Highlighting}
\end{Shaded}

~~

This data set was made by processing data obtained from FBREF.com

~~

\begin{Shaded}
\begin{Highlighting}[]
\NormalTok{xbrks }\OtherTok{\textless{}{-}} \FunctionTok{seq}\NormalTok{(}\SpecialCharTok{{-}}\FloatTok{0.5}\NormalTok{, }\FunctionTok{max}\NormalTok{(xdf[ , }\StringTok{"Gls"}\NormalTok{])}\SpecialCharTok{+}\FloatTok{0.5}\NormalTok{, }\AttributeTok{by=}\DecValTok{1}\NormalTok{)}
\FunctionTok{par}\NormalTok{(}\AttributeTok{cex=}\FloatTok{0.65}\NormalTok{)}
\FunctionTok{hist}\NormalTok{(xdf[ , }\StringTok{"Gls"}\NormalTok{], }\AttributeTok{breaks=}\NormalTok{xbrks, }\AttributeTok{main=}\StringTok{"Team{-}Game Goals Scored, NWSL 2017{-}2021 Season"}\NormalTok{)}
\end{Highlighting}
\end{Shaded}

\begin{figure}
\centering
\includegraphics{Lab_7_X_files/figure-latex/unnamed-chunk-2-1.pdf}
\caption{Distribution Team Goals scored by game.}
\end{figure}

~~

Let's also create a frequency table.

\begin{Shaded}
\begin{Highlighting}[]
\FunctionTok{library}\NormalTok{(xtable)}
\FunctionTok{options}\NormalTok{(}\AttributeTok{xtable.comment =} \ConstantTok{FALSE}\NormalTok{)}

\NormalTok{xtbl }\OtherTok{\textless{}{-}} \FunctionTok{table}\NormalTok{(}\StringTok{"goals"}\OtherTok{=}\NormalTok{xdf[ , }\StringTok{"Gls"}\NormalTok{])}
\NormalTok{xtbl }\OtherTok{\textless{}{-}} \FunctionTok{as.data.frame}\NormalTok{(xtbl)}
\FunctionTok{print}\NormalTok{(}\FunctionTok{xtable}\NormalTok{(xtbl, }\AttributeTok{caption=}\StringTok{"NWSL game{-}team goals scored."}\NormalTok{), }\AttributeTok{include.rownames=}\ConstantTok{FALSE}\NormalTok{)}
\end{Highlighting}
\end{Shaded}

\begin{table}[ht]
\centering
\begin{tabular}{lr}
  \hline
goals & Freq \\ 
  \hline
0 & 312 \\ 
  1 & 390 \\ 
  2 & 212 \\ 
  3 &  92 \\ 
  4 &  38 \\ 
  5 &  11 \\ 
  6 &   3 \\ 
   \hline
\end{tabular}
\caption{NWSL game-team goals scored.} 
\end{table}

~~

Calculate some summaries.

~~

\begin{Shaded}
\begin{Highlighting}[]
\NormalTok{xvarnames }\OtherTok{\textless{}{-}} \FunctionTok{c}\NormalTok{(}\StringTok{"Gls"}\NormalTok{, }\StringTok{"Ast"}\NormalTok{, }\StringTok{"PK"}\NormalTok{, }\StringTok{"PKatt"}\NormalTok{, }\StringTok{"CrdY"}\NormalTok{, }\StringTok{"CrdR"}\NormalTok{)}

\NormalTok{xmins }\OtherTok{\textless{}{-}} \FunctionTok{apply}\NormalTok{(xdf[ , xvarnames], }\DecValTok{2}\NormalTok{, min)}

\NormalTok{xmeans }\OtherTok{\textless{}{-}} \FunctionTok{apply}\NormalTok{(xdf[ , xvarnames], }\DecValTok{2}\NormalTok{, mean)}

\NormalTok{xmedians }\OtherTok{\textless{}{-}} \FunctionTok{apply}\NormalTok{(xdf[ , xvarnames], }\DecValTok{2}\NormalTok{, median)}

\NormalTok{xsds }\OtherTok{\textless{}{-}} \FunctionTok{apply}\NormalTok{(xdf[ , xvarnames], }\DecValTok{2}\NormalTok{, sd)}

\NormalTok{xIQRs }\OtherTok{\textless{}{-}} \FunctionTok{apply}\NormalTok{(xdf[ , xvarnames], }\DecValTok{2}\NormalTok{, IQR)}

\NormalTok{xmaxs }\OtherTok{\textless{}{-}} \FunctionTok{apply}\NormalTok{(xdf[ , xvarnames], }\DecValTok{2}\NormalTok{, max)}

\NormalTok{xsummaries }\OtherTok{\textless{}{-}} \FunctionTok{rbind}\NormalTok{(xmins, xmeans, xmedians, xsds, xIQRs, xmaxs)}

\FunctionTok{rownames}\NormalTok{(xsummaries) }\OtherTok{\textless{}{-}} \FunctionTok{c}\NormalTok{(}\StringTok{"min"}\NormalTok{, }\StringTok{"mean"}\NormalTok{, }\StringTok{"median"}\NormalTok{, }\StringTok{"sd"}\NormalTok{, }\StringTok{"IQR"}\NormalTok{, }\StringTok{"max"}\NormalTok{)}

\FunctionTok{print}\NormalTok{(}\FunctionTok{xtable}\NormalTok{(xsummaries, }\AttributeTok{caption=}\StringTok{"NWSL game{-}team summary statistics."}\NormalTok{), }\AttributeTok{include.rownames=}\ConstantTok{TRUE}\NormalTok{)}
\end{Highlighting}
\end{Shaded}

\begin{table}[ht]
\centering
\begin{tabular}{rrrrrrr}
  \hline
 & Gls & Ast & PK & PKatt & CrdY & CrdR \\ 
  \hline
min & 0.00 & 0.00 & 0.00 & 0.00 & 0.00 & 0.00 \\ 
  mean & 1.24 & 0.85 & 0.08 & 0.12 & 1.05 & 0.03 \\ 
  median & 1.00 & 1.00 & 0.00 & 0.00 & 1.00 & 0.00 \\ 
  sd & 1.16 & 0.98 & 0.29 & 0.35 & 0.96 & 0.19 \\ 
  IQR & 2.00 & 1.00 & 0.00 & 0.00 & 2.00 & 0.00 \\ 
  max & 6.00 & 6.00 & 2.00 & 2.00 & 5.00 & 2.00 \\ 
   \hline
\end{tabular}
\caption{NWSL game-team summary statistics.} 
\end{table}

~~

Let's take a look at average game-team goals by season

~~

\begin{Shaded}
\begin{Highlighting}[]
\NormalTok{xagg }\OtherTok{\textless{}{-}} \FunctionTok{aggregate}\NormalTok{(xdf[ , }\StringTok{"Gls"}\NormalTok{], }\AttributeTok{by=}\FunctionTok{list}\NormalTok{(xdf[ , }\StringTok{"season"}\NormalTok{]), mean)}

\FunctionTok{colnames}\NormalTok{(xagg) }\OtherTok{\textless{}{-}} \FunctionTok{c}\NormalTok{(}\StringTok{"Season"}\NormalTok{, }\StringTok{"Goals"}\NormalTok{)}

\FunctionTok{print}\NormalTok{(}\FunctionTok{xtable}\NormalTok{(xagg, }\AttributeTok{caption=}\StringTok{"NWSL avg game{-}team goals scored by season."}\NormalTok{), }\AttributeTok{include.rownames=}\ConstantTok{FALSE}\NormalTok{)}
\end{Highlighting}
\end{Shaded}

\begin{table}[ht]
\centering
\begin{tabular}{rr}
  \hline
Season & Goals \\ 
  \hline
2017 & 1.39 \\ 
  2018 & 1.27 \\ 
  2019 & 1.26 \\ 
  2020 & 1.18 \\ 
  2021 & 1.10 \\ 
   \hline
\end{tabular}
\caption{NWSL avg game-team goals scored by season.} 
\end{table}

\begin{Shaded}
\begin{Highlighting}[]
\CommentTok{\#x \textless{}{-} as.factor(xagg[ , "Season"])}
\NormalTok{x }\OtherTok{\textless{}{-}}\NormalTok{ xagg[ , }\StringTok{"Season"}\NormalTok{]}
\NormalTok{y }\OtherTok{\textless{}{-}}\NormalTok{ xagg[ , }\StringTok{"Goals"}\NormalTok{]}
\FunctionTok{par}\NormalTok{(}\AttributeTok{cex=}\FloatTok{0.65}\NormalTok{)}
\FunctionTok{plot}\NormalTok{(x, y, }\AttributeTok{type=}\StringTok{"l"}\NormalTok{, }\AttributeTok{col=}\StringTok{"\#3355BB"}\NormalTok{, }\AttributeTok{main=}\StringTok{"Avg Team{-}Game Goals Scored by NWSL Season"}\NormalTok{, }
     \AttributeTok{xaxt=}\StringTok{"n"}\NormalTok{, }\AttributeTok{xlab=}\StringTok{"Season"}\NormalTok{, }\AttributeTok{ylab=}\StringTok{"Avg Team{-}Game Goals"}\NormalTok{, }\AttributeTok{lwd=}\DecValTok{2}\NormalTok{)}
\FunctionTok{axis}\NormalTok{(}\DecValTok{1}\NormalTok{, }\FunctionTok{c}\NormalTok{(}\DecValTok{2017}\NormalTok{, }\DecValTok{2018}\NormalTok{, }\DecValTok{2019}\NormalTok{, }\DecValTok{2020}\NormalTok{, }\DecValTok{2021}\NormalTok{), }\FunctionTok{c}\NormalTok{(}\StringTok{"2017"}\NormalTok{, }\StringTok{"2018"}\NormalTok{, }\StringTok{"2019"}\NormalTok{, }\StringTok{"2020"}\NormalTok{, }\StringTok{"2021"}\NormalTok{))}
\end{Highlighting}
\end{Shaded}

\begin{figure}
\centering
\includegraphics{Lab_7_X_files/figure-latex/unnamed-chunk-6-1.pdf}
\caption{NWSL Avg Team-Game Goals by Season.}
\end{figure}

~~

\hypertarget{simple-random-sampling}{%
\subsubsection{Simple Random Sampling}\label{simple-random-sampling}}

~~

This is an illustration.

We are going to draw 200,000 random samples of size \(n=16\), with
replacement, from our NWSL team-game data, and for each simulation
calculate the sample average. We'll then use a histogram to graphically
convey this empirical sampling distribution.

By the way, when we sample with replacement we are defining our
population as being comprised of an infinite collection of copies of our
data.

We are simulating unbiased sampling --- in particular, simple random
sampling.

~~

\begin{Shaded}
\begin{Highlighting}[]
\FunctionTok{set.seed}\NormalTok{(}\DecValTok{777}\NormalTok{)}

\NormalTok{nn }\OtherTok{\textless{}{-}} \DecValTok{200000}

\NormalTok{N }\OtherTok{\textless{}{-}} \DecValTok{16}

\NormalTok{xavg\_vec }\OtherTok{\textless{}{-}} \FunctionTok{numeric}\NormalTok{(nn)}

\ControlFlowTok{for}\NormalTok{(i }\ControlFlowTok{in} \DecValTok{1}\SpecialCharTok{:}\FunctionTok{length}\NormalTok{(xavg\_vec)) \{}
\NormalTok{    xavg\_vec[i] }\OtherTok{\textless{}{-}} \FunctionTok{mean}\NormalTok{( }\FunctionTok{sample}\NormalTok{(xdf[ , }\StringTok{"Gls"}\NormalTok{], }\AttributeTok{size=}\NormalTok{N, }\AttributeTok{replace=}\ConstantTok{TRUE}\NormalTok{) )}
\NormalTok{\}}
\end{Highlighting}
\end{Shaded}

\begin{Shaded}
\begin{Highlighting}[]
\FunctionTok{par}\NormalTok{(}\AttributeTok{cex=}\FloatTok{0.65}\NormalTok{)}
\FunctionTok{hist}\NormalTok{(xavg\_vec, }\AttributeTok{main=}\StringTok{"Empiric Sampling Dist. NWSL Team{-}Game Goals, n=16"}\NormalTok{)}
\FunctionTok{abline}\NormalTok{(}\AttributeTok{v=}\FunctionTok{mean}\NormalTok{(xavg\_vec), }\AttributeTok{col=}\StringTok{"\#33BB33AA"}\NormalTok{, }\AttributeTok{lwd=}\DecValTok{7}\NormalTok{)}
\FunctionTok{abline}\NormalTok{(}\AttributeTok{v=}\FunctionTok{mean}\NormalTok{(xdf[ , }\StringTok{"Gls"}\NormalTok{]), }\AttributeTok{col=}\StringTok{"\#BB3333AA"}\NormalTok{, }\AttributeTok{lwd=}\DecValTok{3}\NormalTok{)}
\end{Highlighting}
\end{Shaded}

\begin{figure}
\centering
\includegraphics{Lab_7_X_files/figure-latex/esd1-1.pdf}
\caption{Empirical Sampling Distribution of mean Team-Game goals from
NWSL, sample size of 16. The red line marks the true population
parameter mean team-game goals; the green, the mean of all the simulated
sample means.}
\end{figure}

~~

The fact that the red and green lines in Figure @ref(fig:esd1) closely
mark the same value tell us that the population mean and the average of
the sample means closely agree. The suggestion is that the method of
sampling (the sample() function in \textbf{R}), is unbiased.

~~

\hypertarget{deliberately-biased-sampling}{%
\subsubsection{Deliberately Biased
Sampling}\label{deliberately-biased-sampling}}

~~

We can deliberately illustrate bias by explicitly not drawing samples
with impartiality. For example, we can draw team-game goals only from
the 2017 season.

~~

\begin{Shaded}
\begin{Highlighting}[]
\FunctionTok{set.seed}\NormalTok{(}\DecValTok{777}\NormalTok{)}

\NormalTok{nn }\OtherTok{\textless{}{-}} \DecValTok{200000}

\NormalTok{N }\OtherTok{\textless{}{-}} \DecValTok{16}

\NormalTok{xavg\_vec }\OtherTok{\textless{}{-}} \FunctionTok{numeric}\NormalTok{(nn)}

\NormalTok{xbiasmask }\OtherTok{\textless{}{-}}\NormalTok{ xdf[ , }\StringTok{"season"}\NormalTok{] }\SpecialCharTok{\%in\%} \StringTok{"2017"}
\CommentTok{\#xbiasmask \textless{}{-} xdf[ , "team"] \%in\% "Washington Spirit"}

\ControlFlowTok{for}\NormalTok{(i }\ControlFlowTok{in} \DecValTok{1}\SpecialCharTok{:}\FunctionTok{length}\NormalTok{(xavg\_vec)) \{}
\NormalTok{    xavg\_vec[i] }\OtherTok{\textless{}{-}} \FunctionTok{mean}\NormalTok{( }\FunctionTok{sample}\NormalTok{(xdf[ xbiasmask, }\StringTok{"Gls"}\NormalTok{], }\AttributeTok{size=}\NormalTok{N, }\AttributeTok{replace=}\ConstantTok{TRUE}\NormalTok{) )}
\NormalTok{\}}
\end{Highlighting}
\end{Shaded}

\begin{Shaded}
\begin{Highlighting}[]
\FunctionTok{par}\NormalTok{(}\AttributeTok{cex=}\FloatTok{0.65}\NormalTok{)}
\FunctionTok{hist}\NormalTok{(xavg\_vec, }\AttributeTok{main=}\StringTok{"Biased Empiric Sampling Dist. NWSL Team{-}Game Goals, n=16"}\NormalTok{)}
\FunctionTok{abline}\NormalTok{(}\AttributeTok{v=}\FunctionTok{mean}\NormalTok{(xavg\_vec), }\AttributeTok{col=}\StringTok{"\#33BB33AA"}\NormalTok{, }\AttributeTok{lwd=}\DecValTok{7}\NormalTok{)}
\FunctionTok{abline}\NormalTok{(}\AttributeTok{v=}\FunctionTok{mean}\NormalTok{(xdf[ , }\StringTok{"Gls"}\NormalTok{]), }\AttributeTok{col=}\StringTok{"\#BB3333AA"}\NormalTok{, }\AttributeTok{lwd=}\DecValTok{3}\NormalTok{)}
\end{Highlighting}
\end{Shaded}

\begin{figure}
\centering
\includegraphics{Lab_7_X_files/figure-latex/esd2-1.pdf}
\caption{Empirical Sampling Distribution of mean Team-Game goals from
NWSL, sample size of 16, where sample was drawn only from 2017 season.
The red line marks the true population parameter mean team-game goals;
the green, the mean of all the simulated sample means.}
\end{figure}

~~

The fact that the red and green lines in Figure @ref(fig:esd2) do not
mark the same value tell us that the population mean and the average of
the sample means are different. The suggestion is that the method of
sampling (only drawing from one season) is \textbf{biased}.

\newpage

\hypertarget{confidence-interval-for-population-proportion}{%
\subsection{Confidence Interval for Population
Proportion}\label{confidence-interval-for-population-proportion}}

~~

\hypertarget{super-bowl-coin-tosses}{%
\subsubsection{Super Bowl Coin Tosses}\label{super-bowl-coin-tosses}}

\begin{Shaded}
\begin{Highlighting}[]
\DocumentationTok{\#\# Read in our data}
\NormalTok{sbdf }\OtherTok{\textless{}{-}} \FunctionTok{read.table}\NormalTok{(}\StringTok{"SuperBowl\_coinTosses.tsv"}\NormalTok{, }\AttributeTok{sep=}\StringTok{"}\SpecialCharTok{\textbackslash{}t}\StringTok{"}\NormalTok{, }\AttributeTok{header=}\ConstantTok{TRUE}\NormalTok{)}

\FunctionTok{head}\NormalTok{(sbdf, }\AttributeTok{n=}\DecValTok{6}\NormalTok{)}
\end{Highlighting}
\end{Shaded}

\begin{verbatim}
##   SuperBowl             Matchup CoinToss Coin.TossWinner GameWinner
## 1         1   Chiefs vs Packers    Heads         Packers    Packers
## 2         2  Packers vs Raiders    Tails         Raiders    Packers
## 3         3       Colts vs Jets    Heads            Jets       Jets
## 4         4   Vikings vs Chiefs    Tails         Vikings     Chiefs
## 5         5    Colts vs Cowboys    Tails         Cowboys      Colts
## 6         6 Cowboys vs Dolphins    Heads        Dolphins    Cowboys
##   CoinTossWinnerEqualGameWinner.
## 1                            Yes
## 2                             No
## 3                            Yes
## 4                             No
## 5                             No
## 6                             No
\end{verbatim}

~~

This data set was obtained from
\url{https://www.sportsbettingdime.com/guides/resources/super-bowl-coin-toss-history/}

~~

Imagine that these 56 observations were randomly realized from an
infinite population of Super Bowl coin tosses.

Let's create a 68\% Confidence Interval for the true, unknown population
parameter proportion of Heads.

\begin{Shaded}
\begin{Highlighting}[]
\NormalTok{n }\OtherTok{\textless{}{-}} \FunctionTok{nrow}\NormalTok{(sbdf)}

\NormalTok{xheads }\OtherTok{\textless{}{-}} \FunctionTok{as.integer}\NormalTok{(sbdf[, }\StringTok{"CoinToss"}\NormalTok{] }\SpecialCharTok{\%in\%} \StringTok{"Heads"}\NormalTok{)}

\NormalTok{phat }\OtherTok{\textless{}{-}} \FunctionTok{mean}\NormalTok{(xheads)}
\NormalTok{phat}
\end{Highlighting}
\end{Shaded}

\begin{verbatim}
## [1] 0.4821429
\end{verbatim}

\begin{Shaded}
\begin{Highlighting}[]
\DocumentationTok{\#\#\# OR}

\NormalTok{phat }\OtherTok{\textless{}{-}} \FunctionTok{sum}\NormalTok{(xheads) }\SpecialCharTok{/} \FunctionTok{length}\NormalTok{(xheads)}
\NormalTok{phat}
\end{Highlighting}
\end{Shaded}

\begin{verbatim}
## [1] 0.4821429
\end{verbatim}

\begin{Shaded}
\begin{Highlighting}[]
\NormalTok{var\_phat }\OtherTok{\textless{}{-}}\NormalTok{ phat }\SpecialCharTok{*}\NormalTok{ (}\DecValTok{1} \SpecialCharTok{{-}}\NormalTok{ phat)}
\NormalTok{var\_phat}
\end{Highlighting}
\end{Shaded}

\begin{verbatim}
## [1] 0.2496811
\end{verbatim}

\begin{Shaded}
\begin{Highlighting}[]
\NormalTok{est\_SE }\OtherTok{\textless{}{-}} \FunctionTok{sqrt}\NormalTok{(var\_phat }\SpecialCharTok{/}\NormalTok{ n)}
\NormalTok{est\_SE}
\end{Highlighting}
\end{Shaded}

\begin{verbatim}
## [1] 0.06677269
\end{verbatim}

~~

We need to calculate our estimated margin of error from our confidence
level.

We're using the normal approximation. We know from the empiric rule that
\(-1 < Z < 1\) chops out about the middle 68\% of the normal
distribution.

We can also have \textbf{R} do this for us.

~~

\begin{Shaded}
\begin{Highlighting}[]
\NormalTok{z\_low }\OtherTok{\textless{}{-}} \FunctionTok{qnorm}\NormalTok{( (}\DecValTok{1} \SpecialCharTok{{-}} \FloatTok{0.68}\NormalTok{) }\SpecialCharTok{/} \DecValTok{2}\NormalTok{ )}
\NormalTok{z\_low}
\end{Highlighting}
\end{Shaded}

\begin{verbatim}
## [1] -0.9944579
\end{verbatim}

\begin{Shaded}
\begin{Highlighting}[]
\NormalTok{z\_high }\OtherTok{\textless{}{-}} \FunctionTok{qnorm}\NormalTok{( }\FloatTok{0.68} \SpecialCharTok{+}\NormalTok{ (}\DecValTok{1} \SpecialCharTok{{-}} \FloatTok{0.68}\NormalTok{) }\SpecialCharTok{/} \DecValTok{2}\NormalTok{ )}
\NormalTok{z\_high}
\end{Highlighting}
\end{Shaded}

\begin{verbatim}
## [1] 0.9944579
\end{verbatim}

~~

Let's just use \(z_{L} = -1\) and \(z_{H} = 1\)

~~

\begin{Shaded}
\begin{Highlighting}[]
\NormalTok{z\_low }\OtherTok{\textless{}{-}} \SpecialCharTok{{-}}\DecValTok{1}
\NormalTok{z\_high }\OtherTok{\textless{}{-}} \DecValTok{1}

\NormalTok{CI\_low }\OtherTok{\textless{}{-}}\NormalTok{ phat }\SpecialCharTok{+}\NormalTok{ z\_low }\SpecialCharTok{*}\NormalTok{ est\_SE}

\NormalTok{CI\_high }\OtherTok{\textless{}{-}}\NormalTok{ phat }\SpecialCharTok{+}\NormalTok{ z\_high }\SpecialCharTok{*}\NormalTok{ est\_SE}
\end{Highlighting}
\end{Shaded}

~~

We are 68\% confident that the true proportion of heads within this
imaginary, infinite population of Super Bowl coin tosses is between
0.4154 and 0.5489.

\newpage

\hypertarget{your-work}{%
\section{Your Work}\label{your-work}}

Make sure to edit the ``author'' information in the YAML header near the
top to include your name and UID.

Complete/answer the following.

1 --- Consider Figure @ref(fig:esd2). Repeat the process we used to
illustrate biased sampling, except instead of drawing only from one
season, draw your sample from all seasons but only rows where the team
is ``Washington Spirit''. Comment on your findings.

\begin{Shaded}
\begin{Highlighting}[]
\FunctionTok{set.seed}\NormalTok{(}\DecValTok{777}\NormalTok{)}
\NormalTok{nn }\OtherTok{\textless{}{-}} \DecValTok{200000}
\NormalTok{N }\OtherTok{\textless{}{-}} \DecValTok{16}
\NormalTok{xavg\_vec }\OtherTok{\textless{}{-}} \FunctionTok{numeric}\NormalTok{(nn)}
\NormalTok{xbiasmask }\OtherTok{\textless{}{-}}\NormalTok{ xdf[ , }\StringTok{"team"}\NormalTok{] }\SpecialCharTok{\%in\%} \StringTok{"Washington Spirit"}
\ControlFlowTok{for}\NormalTok{(i }\ControlFlowTok{in} \DecValTok{1}\SpecialCharTok{:}\FunctionTok{length}\NormalTok{(xavg\_vec)) \{}
\NormalTok{xavg\_vec[i] }\OtherTok{\textless{}{-}} \FunctionTok{mean}\NormalTok{( }\FunctionTok{sample}\NormalTok{(xdf[ xbiasmask, }\StringTok{"Gls"}\NormalTok{], }\AttributeTok{size=}\NormalTok{N, }\AttributeTok{replace=}\ConstantTok{TRUE}\NormalTok{) )}
\NormalTok{\}}

\FunctionTok{par}\NormalTok{(}\AttributeTok{cex=}\FloatTok{0.65}\NormalTok{)}
\FunctionTok{hist}\NormalTok{(xavg\_vec, }\AttributeTok{main=}\StringTok{"Biased Empiric Sampling Dist. NWSL Team{-}Game Goals, n=16"}\NormalTok{)}
\FunctionTok{abline}\NormalTok{(}\AttributeTok{v=}\FunctionTok{mean}\NormalTok{(xavg\_vec), }\AttributeTok{col=}\StringTok{"\#33BB33AA"}\NormalTok{, }\AttributeTok{lwd=}\DecValTok{7}\NormalTok{)}
\FunctionTok{abline}\NormalTok{(}\AttributeTok{v=}\FunctionTok{mean}\NormalTok{(xdf[ , }\StringTok{"Gls"}\NormalTok{]), }\AttributeTok{col=}\StringTok{"\#BB3333AA"}\NormalTok{, }\AttributeTok{lwd=}\DecValTok{3}\NormalTok{)}
\end{Highlighting}
\end{Shaded}

\includegraphics{Lab_7_X_files/figure-latex/unnamed-chunk-13-1.pdf}

~

2 --- Show that we have met the requirements for using the normal
approximation to the binomial model in our CI calculation above.

\begin{Shaded}
\begin{Highlighting}[]
\CommentTok{\#To show that the binomial distribution is correct, it must meet the following criteria. }
\CommentTok{\#np ≥ 5}
\CommentTok{\#n(1{-}p) ≥ 5}

\CommentTok{\# xdf \textless{}{-} read.table("NWSL\_gameTeam.tsv", sep="\textbackslash{}t", header=TRUE)}
\CommentTok{\# n \textless{}{-} nrow(xdf)}
\CommentTok{\# }
\CommentTok{\# xheads \textless{}{-} as.integer(xdf[, "team"] \%in\% "Washington Spirit")}
\CommentTok{\# }
\CommentTok{\# phat \textless{}{-} mean(xheads)}
\CommentTok{\# phat}
\CommentTok{\# }
\CommentTok{\# }\AlertTok{\#\#\#}\CommentTok{ OR}
\CommentTok{\# }
\CommentTok{\# phat \textless{}{-} sum(xheads) / length(xheads)}
\CommentTok{\# phat}
\CommentTok{\# n* phat}
\CommentTok{\# n* (1{-}phat)}

\CommentTok{\# As phat is from the binomial distribution, and both of these are greater than 10, these have met the requirements listed above}
\end{Highlighting}
\end{Shaded}

~

3 --- Imagine our NWSL data set is actually a randomly realized
collection of games from an infinite collection of games that could have
resulted. Thinking about the population parameter proportion of the
occurrence of a team drawing one or more Red Cards in a game, create a
90\% CI, a 95\% CI, and a 99\% CI. Comment and interpret your results.
Also show that we have met the requirements for using the normal
approximation to the binomial model.

\begin{Shaded}
\begin{Highlighting}[]
\DocumentationTok{\#\#\# here\textquotesingle{}s a head start}

\NormalTok{xteamGame\_redCard }\OtherTok{\textless{}{-}} \FunctionTok{as.integer}\NormalTok{( xdf[ , }\StringTok{"CrdR"}\NormalTok{ ] }\SpecialCharTok{\textgreater{}} \DecValTok{0}\NormalTok{ )}

\NormalTok{phat\_rc }\OtherTok{\textless{}{-}} \FunctionTok{mean}\NormalTok{(xteamGame\_redCard)}
\NormalTok{phat\_rc}
\end{Highlighting}
\end{Shaded}

\begin{verbatim}
## [1] 0.03213611
\end{verbatim}

\begin{Shaded}
\begin{Highlighting}[]
\NormalTok{est\_SE\_rc }\OtherTok{\textless{}{-}} \FunctionTok{sqrt}\NormalTok{( phat\_rc }\SpecialCharTok{*}\NormalTok{ (}\DecValTok{1} \SpecialCharTok{{-}}\NormalTok{ phat\_rc) }\SpecialCharTok{/} \FunctionTok{length}\NormalTok{(xteamGame\_redCard) ) }
\NormalTok{est\_SE\_rc}
\end{Highlighting}
\end{Shaded}

\begin{verbatim}
## [1] 0.005422018
\end{verbatim}

\end{document}
